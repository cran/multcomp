\documentclass{article}
\usepackage{amsmath}
\usepackage[round]{natbib}
\renewcommand{\baselinestretch}{1.5}
%%\thispagestyle{empty}

\newcommand{\bbeta}{\mbox{\boldmath $\beta$}}
\newcommand{\bepsilon}{\mbox{\boldmath $\epsilon$}}
\newcommand{\bc}{{\mathbf{c}}}
\newcommand{\bC}{{\mathbf{C}}}
\newcommand{\bD}{{\mathbf{D}}}
\newcommand{\bI}{{\mathbf{I}}}
\newcommand{\bR}{{\mathbf{R}}}
\newcommand{\bX}{{\mathbf{X}}}
\newcommand{\bY}{{\mathbf{Y}}}
\newcommand{\bzero}{{\mathbf{0}}}

%%\VignetteIndexEntry{How to do multiple comparisions with the multcomp package}
%%\VignetteDepends{multcomp}


\usepackage{/usr/local/lib/R/share/texmf/Sweave}
\begin{document}



\title{On Multiple Comparisons in \textsf{R} \thanks{This document is based
on an article to appear in \textsf{R} News} }
\author{by Frank Bretz, Torsten Hothorn and Peter Westfall}
\date{}
\maketitle


\section{Description}

The multiplicity problem arises when several inferences are considered
simultaneously as a group.  If each inference has a $5\%$ error rate, 
then the error rate over the entire group can be much higher than $5\%$.  
This article shows practical examples of multiple comparisons 
procedures that control the error of making any incorrect inference.

The \texttt{multcomp} package for the \textsf{R} statistical
environment allows for multiple comparisons of parameters whose
estimates are generally correlated, including comparisons of $k$
groups in general linear models. The package has many common
multiple comparison procedures ``hard-coded'', including Dunnett,
Tukey, sequential pairwise contrasts, comparisons with the
average, changepoint analysis, Williams', Marcus', McDermott's,
and tetrad contrasts. In addition, a free input interface for the
contrast matrix allows for more general comparisons.

The comparisons themselves are not restricted to balanced or simple
designs. Instead, the package is designed to provide general
multiple comparisons, thus allowing for covariates, nested
effects, correlated means, likelihood-based estimates, and missing
values. For the homoscedastic normal linear models, the functions in the package
account for the correlations between test statistics by using the
exact multivariate $t$-distribution. The resulting procedures are
therefore more powerful than the Bonferroni and Holm methods;
adjusted p-values for these methods are reported for reference.
For more general models,  the program accounts for correlations
using the asymptotic multivariate normal distribution; examples
include multiple comparisons based on rank transformations,
logistic regression, GEEs, and proportional hazards models.  In
the asymptotic case, the user must supply the estimates, the
asymptotic covariance matrix, and the contrast matrix.

Basically, the package provides two functions. The first, \texttt{simint}, computes 
confidence
intervals for the common single-step procedures.
This approach is uniformly improved by the second function
(\texttt{simtest}), which utilizes logical constraints and is
closely related to closed testing. However, no confidence
intervals are available for the \texttt{simtest} function. 
For testing and validation purposes, some 
examples from \cite{multiple-c:1999} are included in the package.

\section{Details}

Assume the general linear model
$$
\bY = \bX \bbeta + \bepsilon,
$$
where $\bY$ is the $n \times 1$ observation vector, $\bX$ is the
fixed and known $n\times p$ design matrix, $\bbeta$ is the fixed
and unknown $p \times 1$ parameter vector and $\bepsilon$ is the
random, unobservable $n \times 1$ error vector, distributed as
$N_n(\bzero, \sigma^2\bI_n)$. We assume the usual estimates
$$
\hat{\bbeta} = (\bX^t \bX)^-\bX^t\bY
$$
and
$$
\hat{\sigma}^2 = (\bY-\bX\hat{\bbeta})^t(\bY-\bX\hat{\bbeta})/\nu,
$$
where $\nu = n - \mbox{rank}(\bX)$. Our focus is on multiple
comparisons for parameters of the general form $\bc^t \bbeta$.
Its variance is given through
$$
\mbox{Var}(\bc^t \hat{\bbeta}) = \hat{\sigma}^2\bc^t(\bX^t \bX)^-\bc.
$$

In simultaneous inferences we are faced with a given family of
estimable parameters $\{\bc_1^t \bbeta, \ldots, \bc_k^t \bbeta\}$.
We thus use the pivotal test statistics
$$
T_i = \frac{\bc_i^t \hat{\bbeta}-\bc_i^t
\bbeta}{\hat{\sigma}\sqrt{\bc_i^t(\bX^t \bX)^-\bc_i}}.
$$
For a general account on multiple comparison procedures we refer
to \cite{HochbergTamhane:1987}. The joint distribution of $\{T_1,
\ldots, T_k\}$ is multivariate $t$ with degrees of freedom $\nu$
and correlation matrix $\bR = \bD\bC(\bX^t \bX)^-\bC^t\bD$, where
$\bC^t = (c_1, \ldots, c_k)$ and $\bD = \mbox{diag}(\bc_i^t(\bX^t
\bX)^-\bc_i)^{-1/2}$. In the asymptotic case $\nu \rightarrow
\infty$ or if $\sigma$ is known, the corresponding limiting
multivariate normal distribution holds. The numerical evaluation
of the multivariate $t$ and normal distribution is available with
the \textsf{R} package \texttt{mvtnorm}, see \cite{hothornetal:2001}.

The function \texttt{simint} provides simultaneous confidence
intervals for the estimable functions $\bc_i^t \bbeta$ in the
(two-sided) form
$$
\left[ \bc_i^t \hat{\bbeta}-c_{1-\alpha}\hat{\sigma}\sqrt{\bc_i^t(\bX^t
\bX)^-\bc_i};
\bc_i^t \hat{\bbeta}+c_{1-\alpha}\hat{\sigma}\sqrt{\bc_i^t(\bX^t
\bX)^-\bc_i} \right],
$$
where $c_{1-\alpha}$ is the critical value at level $1-\alpha$, as
derived under the distributional assumptions above. If lower or
upper tailed tests are used, the corresponding interval bounds are
set to $-\infty$ and $\infty$, respectively.

The second function \texttt{simtest} provides more powerful test
decisions than \texttt{simint} yet it does not provide
simultaneous confidence intervals. It uses the stepwise methods of
\cite{westfall:1997}, which take the logical constraints between the
hypotheses into account and which are closely related to the
closed testing principle of \cite{Marcusetal:1976}. In addition, the
stochastic dependencies of the test statistics are incorporated,
thus allowing imbalance, covariates and more general models.
Again, any collection of linear combinations of the estimable
parameters is allowed, not just pairwise comparisons. We refer to
\cite{westfall:1997} for the algebraic and algorithmic details.

\section{Example}

We illustrate some of the capabilities of the \texttt{multcomp}
package using the \texttt{recovery} dataset. Three
different heating blankets $b_1, b_2, b_3$ for post-surgery
treatment are compared to a standard blanket $b_0$. The variable
of interest in this simple one-way layout was recovery time in
minutes of patients allocated randomly to one of the four
treatments. The standard approach for comparing several treatments
against a control is the many-to-one test of Dunnett (1955). The
Dunnett test is one of the ``hard-coded'' procedures available for
one-factor models in \texttt{multcomp}. To obtain simultaneous
confidence intervals for the comparisons $\beta_i - \beta_1$ on
simply calls:
\small
\begin{Schunk}
\begin{Sinput}
>library(multcomp)
>data(recovery)
>Dcirec <- simint(minutes ~ blanket, data = recovery, 
+     conf.level = 0.9, alternative = "less")
\end{Sinput}
\end{Schunk}
\begin{Schunk}
\begin{Sinput}
>print(Dcirec)
\end{Sinput}
\begin{Soutput}
	Simultaneous confidence intervals: Dunnett
	contrasts

	90 % confidence intervals

                    Estimate lower CI upper CI
blanketb1-blanketb0   -2.133     -Inf    0.823
blanketb2-blanketb0   -7.467     -Inf   -4.511
blanketb3-blanketb0   -1.667     -Inf   -0.036
\end{Soutput}
\end{Schunk}
\normalsize
Thus, blankets $b_2$ and $b_3$ lead to significant lower recovery
times in comparison to the standard $b_0$, since the respective
upper confidence bounds are less than 0. In particular, the output
above indicates that at the designated confidence level of 90\%
the average recovery time for $b_2$ is more than 7 minutes shorter
than it is for $b_0$.

A second way to obtain the same results is to define the contrast
matrix $\bC$ explicitly:
\small
\begin{Schunk}
\begin{Sinput}
>C <- matrix(c(0, 0, 0, -1, -1, -1, 1, 0, 0, 0, 
+     1, 0, 0, 0, 1), nc = 5)
>rownames(C) <- paste("C", 1:nrow(C), sep = "")
>Ccirec <- simint(minutes ~ blanket, data = recovery, 
+     conf.level = 0.9, alternative = "less", eps = 1e-04, 
+     cmatrix = C)
\end{Sinput}
\end{Schunk}
\begin{Schunk}
\begin{Sinput}
>print(Ccirec)
\end{Sinput}
\begin{Soutput}
	Simultaneous confidence intervals: user-defined
	contrasts

	90 % confidence intervals

   Estimate lower CI upper CI
C1  -2.1333     -Inf   0.8226
C2  -7.4667     -Inf  -4.5107
C3  -1.6667     -Inf  -0.0359
\end{Soutput}
\end{Schunk}
\normalsize
The first column of $\bC$ stands for the intercept $\beta_0$, the
remaining columns are reserved for the 4 levels $\beta_1, \dots,
\beta_4$ of the single factor. Each row defines a particular
linear combination $\bc_i^t \bbeta$. Note that the {\em eps} argument
specifies the accuracy of the numerical results (see \texttt{pmvt} in package
\texttt{mvtnorm} for more details). This is the reason why
the confidence bounds are now printed with four significant digits 
instead of the former three digits.

More detailed output is available by using the summary method:
\small
\begin{Schunk}
\begin{Sinput}
>summary(Ccirec)
\end{Sinput}
\begin{Soutput}
	Simultaneous 90% confidence intervals: user-defined
	contrasts

	 user-defined contrasts for factor blanket

Contrast matrix:
   [,1] [,2] [,3] [,4] [,5]
C1    0   -1    1    0    0
C2    0   -1    0    1    0
C3    0   -1    0    0    1

Absolute Error Tolerance:  1e-04 

 90 % quantile:  1.8431 

Coefficients:
   Estimate low CI,  upp CI t value Std.Err.  p raw p Bonf
C1  -2.1333    -Inf  0.8226 -1.3302   1.6038 0.0958 0.2874
C2  -7.4667    -Inf -4.5107 -4.6556   1.6038 0.0000 0.0001
C3  -1.6667    -Inf -0.0359 -1.8837   0.8848 0.0337 0.1012
    p adj
C1 0.2412
C2 0.0001
C3 0.0924
\end{Soutput}
\end{Schunk}
\normalsize
This output prints the user defined contrast matrix $\bC$ 
and the quantile $c_{1-\alpha}$. In addition,
simultaneous confidence intervals, the estimates $\bc_i^t
\hat{\bbeta}$ and their standard errors are given as well as the raw
p-values (computed from the marginal $t$ distributions) and
multiplicity adjusted p-values (using either the multivariate $t$ distribution or
the Bonferroni correction). The simultaneous confidence intervals
and the adjusted p-values based on the multivariate $t$ distribution are compatible in
the sense that if $p_{adj}<0.05$, then the associated confidence
interval does not contain the 0.

A more powerful approach is available using the \texttt{simtest}
function. The call remains essentially the same, also no simultaneous confidence intervals are available:
\small
\begin{Schunk}
\begin{Sinput}
>Ctrec <- simtest(minutes ~ blanket, data = recovery, 
+     conf.level = 0.9, alternative = "less", eps = 1e-04, 
+     cmatrix = C)
\end{Sinput}
\end{Schunk}
\begin{Schunk}
\begin{Sinput}
>summary(Ctrec)
\end{Sinput}
\begin{Soutput}
	 Simultaneous tests: user-defined contrasts 

	 user-defined contrasts for factor blanket

Contrast matrix:
   [,1] [,2] [,3] [,4] [,5]
C1    0   -1    1    0    0
C2    0   -1    0    1    0
C3    0   -1    0    0    1


Absolute Error Tolerance:  1e-04 

Coefficients:
   Estimate t value Std.Err.  p raw p Bonf  p adj
C2  -7.4667 -4.6556   1.6038 0.0000 0.0001 0.0001
C3  -1.6667 -1.8837   1.6038 0.0337 0.0675 0.0640
C1  -2.1333 -1.3302   0.8848 0.0958 0.0958 0.0958
\end{Soutput}
\end{Schunk}
\normalsize
It transpires that the adjusted p-values are indeed
uniformly lower in comparison to those from \texttt{simint}.

A final example call illustrates the use of the \texttt{multcomp} package, if the
estimates $\hat{\beta_i}$ and their covariances are passed by
hand. In such cases, the core functions \texttt{csimint} and
\texttt{csimtest} have to be called without using the
\texttt{sim\{int,test\}} interfaces. The call 
\small
\begin{Schunk}
\begin{Sinput}
>parm <- c(14.8, 12.6667, 7.3333, 13.1333)
>N <- c(20, 3, 3, 15)
>contrast <- contrMat(N, type = "Dunnett")
>nu <- 37
>mse <- 6.7099
>covm <- mse * diag(1/N)
>csimint(estpar = parm, df = as.integer(nu), covm = covm, 
+     cmatrix = contrast, conf.level = 0.9, alternative = "less")
\end{Sinput}
\begin{Soutput}
	Simultaneous confidence intervals: user-defined
	contrasts

	90 % confidence intervals

    Estimate lower CI upper CI
2-1   -2.133     -Inf    0.823
3-1   -7.467     -Inf   -4.511
4-1   -1.667     -Inf   -0.036
\end{Soutput}
\end{Schunk}
\normalsize 
yields the same result as the first call above. The
sample size vector $N$ and the mean square error $mse$ are only
required for a convenient computation of the covariance matrix.
Note that the contrast matrix can either be entered by hand or by
using the availability of standard contrast matrices in the
\texttt{contrMat} function.

\section{Graphical Representation}

The method \texttt{plot.hmtest} is available for a graphical inspectation of
the simultaneous confidence intervals. For each contrast, the confidence
interval is plotted, for example \texttt{plot(Dcirec)} can be used for
plotting the one-sided Dunnett confidence intervals for the \texttt{recovery}
example from the first code snippet.
\begin{figure}
\begin{center}
\includegraphics{Rmc-dunnett-plots}
\caption{A graphical representation of one-sided Dunnett confidence
intervals. The intervals are plotted as horizontal lines where the 
limits of the intervals are given by round brackets and the
estimates by a point.}
\end{center}
\end{figure}

\section{Conclusion} 
This article addressed the application of 
multiple comparisons using the \texttt{multcomp} package. The 
present methods cover several standard test procedures and allow 
for user specified type of comparisons. Also the discussion  has 
been devoted to general linear models, the package is also 
applicable to more general linear and nonlinear mixed models as 
long as the covariances between the estimates are known.

Currently, the quantiles of the multivariate $t$ or normal distribution are
computed using \texttt{uniroot} on the p-value functions. This is time
consuming and will be improved in future versions of the \texttt{mvtnorm}
package.

We would like to thank Doug Bates for corrections and
suggestions improving the readability.

\bibliographystyle{plainnat}
\bibliography{rmc}

\end{document}
