\documentclass{article}
\usepackage{amsmath}
\usepackage[round]{natbib}
\renewcommand{\baselinestretch}{1.5}
%%\thispagestyle{empty}

\newcommand{\bbeta}{\mbox{\boldmath $\beta$}}
\newcommand{\bepsilon}{\mbox{\boldmath $\epsilon$}}
\newcommand{\bc}{{\mathbf{c}}}
\newcommand{\bC}{{\mathbf{C}}}
\newcommand{\bD}{{\mathbf{D}}}
\newcommand{\bI}{{\mathbf{I}}}
\newcommand{\bR}{{\mathbf{R}}}
\newcommand{\bX}{{\mathbf{X}}}
\newcommand{\bY}{{\mathbf{Y}}}
\newcommand{\bzero}{{\mathbf{0}}}

%%\VignetteIndexEntry{How to do multiple comparisions with the multcomp package}
%%\VignetteDepends{multcomp}


\usepackage{/usr/local/lib/R/share/texmf/Sweave}
\begin{document}



\title{On Multiple Comparisons in \textsf{R} \thanks{This document is based
on an article to appear in \textsf{R} News} }
\author{by Frank Bretz, Torsten Hothorn and Peter Westfall}
\date{}
\maketitle


\section{Description}

The multiplicity problem arises when several inferences are considered
simultaneously as a group.  If each inference has a $5\%$ error rate, 
then the error rate over the entire group can be much higher than $5\%$.  
This article shows practical examples of multiple comparisons 
procedures that control the error of making any incorrect inference.

The \texttt{multcomp} package for the \textsf{R} statistical
environment allows for multiple comparisons of parameters whose
estimates are generally correlated, including comparisons of $k$
groups in general linear models. The package has many common
multiple comparison procedures ``hard-coded'', including Dunnett,
Tukey, sequential pairwise contrasts, comparisons with the
average, changepoint analysis, Williams', Marcus', McDermott's,
and tetrad contrasts. In addition, a free input interface for the
contrast matrix allows for more general comparisons.

The comparisons themselves are not restricted to balanced or simple
designs. Instead, the package is designed to provide general
multiple comparisons, thus allowing for covariates, nested
effects, correlated means, likelihood-based estimates, and missing
values. For the homoscedastic normal linear models, the functions in the package
account for the correlations between test statistics by using the
exact multivariate $t$-distribution. The resulting procedures are
therefore more powerful than the Bonferroni and Holm methods;
adjusted p-values for these methods are reported for reference.
For more general models,  the program accounts for correlations
using the asymptotic multivariate normal distribution; examples
include multiple comparisons based on rank transformations,
logistic regression, GEEs, and proportional hazards models.  In
the asymptotic case, the user must supply the estimates, the
asymptotic covariance matrix, and the contrast matrix.

Basically, the package provides two functions. The first, \texttt{simint}, computes 
confidence
intervals for the common single-step procedures.
This approach is uniformly improved by the second function
(\texttt{simtest}), which utilizes logical constraints and is
closely related to closed testing. However, no confidence
intervals are available for the \texttt{simtest} function. 
For testing and validation purposes, some 
examples from \cite{multiple-c:1999} are included in the package.

\section{Details}

Assume the general linear model
$$
\bY = \bX \bbeta + \bepsilon,
$$
where $\bY$ is the $n \times 1$ observation vector, $\bX$ is the
fixed and known $n\times p$ design matrix, $\bbeta$ is the fixed
and unknown $p \times 1$ parameter vector and $\bepsilon$ is the
random, unobservable $n \times 1$ error vector, distributed as
$N_n(\bzero, \sigma^2\bI_n)$. We assume the usual estimates
$$
\hat{\bbeta} = (\bX^t \bX)^-\bX^t\bY
$$
and
$$
\hat{\sigma}^2 = (\bY-\bX\hat{\bbeta})^t(\bY-\bX\hat{\bbeta})/\nu,
$$
where $\nu = n - \mbox{rank}(\bX)$. Our focus is on multiple
comparisons for parameters of the general form $\bc^t \bbeta$.
Its variance is given through
$$
\mbox{Var}(\bc^t \hat{\bbeta}) = \hat{\sigma}^2\bc^t(\bX^t \bX)^-\bc.
$$

In simultaneous inferences we are faced with a given family of
estimable parameters $\{\bc_1^t \bbeta, \ldots, \bc_k^t \bbeta\}$.
We thus use the pivotal test statistics
$$
T_i = \frac{\bc_i^t \hat{\bbeta}-\bc_i^t
\bbeta}{\hat{\sigma}\sqrt{\bc_i^t(\bX^t \bX)^-\bc_i}}.
$$
For a general account on multiple comparison procedures we refer
to \cite{HochbergTamhane:1987}. The joint distribution of $\{T_1,
\ldots, T_k\}$ is multivariate $t$ with degrees of freedom $\nu$
and correlation matrix $\bR = \bD\bC(\bX^t \bX)^-\bC^t\bD$, where
$\bC^t = (c_1, \ldots, c_k)$ and $\bD = \mbox{diag}(\bc_i^t(\bX^t
\bX)^-\bc_i)^{-1/2}$. In the asymptotic case $\nu \rightarrow
\infty$ or if $\sigma$ is known, the corresponding limiting
multivariate normal distribution holds. The numerical evaluation
of the multivariate $t$ and normal distribution is available with
the \textsf{R} package \texttt{mvtnorm}, see \cite{hothornetal:2001}.

The function \texttt{simint} provides simultaneous confidence
intervals for the estimable functions $\bc_i^t \bbeta$ in the
(two-sided) form
$$
\left[ \bc_i^t \hat{\bbeta}-c_{1-\alpha}\hat{\sigma}\sqrt{\bc_i^t(\bX^t
\bX)^-\bc_i};
\bc_i^t \hat{\bbeta}+c_{1-\alpha}\hat{\sigma}\sqrt{\bc_i^t(\bX^t
\bX)^-\bc_i} \right],
$$
where $c_{1-\alpha}$ is the critical value at level $1-\alpha$, as
derived under the distributional assumptions above. If lower or
upper tailed tests are used, the corresponding interval bounds are
set to $-\infty$ and $\infty$, respectively.

The second function \texttt{simtest} provides more powerful test
decisions than \texttt{simint} yet it does not provide
simultaneous confidence intervals. It uses the stepwise methods of
\cite{westfall:1997}, which take the logical constraints between the
hypotheses into account and which are closely related to the
closed testing principle of \cite{Marcusetal:1976}. In addition, the
stochastic dependencies of the test statistics are incorporated,
thus allowing imbalance, covariates and more general models.
Again, any collection of linear combinations of the estimable
parameters is allowed, not just pairwise comparisons. We refer to
\cite{westfall:1997} for the algebraic and algorithmic details.

\section{Example}

We illustrate some of the capabilities of the \texttt{multcomp}
package using the \texttt{recovery} dataset. Three
different heating blankets $b_1, b_2, b_3$ for post-surgery
treatment are compared to a standard blanket $b_0$. The variable
of interest in this simple one-way layout was recovery time in
minutes of patients allocated randomly to one of the four
treatments. The standard approach for comparing several treatments
against a control is the many-to-one test of Dunnett (1955). The
Dunnett test is one of the ``hard-coded'' procedures available for
one-factor models in \texttt{multcomp}. To obtain simultaneous
confidence intervals for the comparisons $\beta_i - \beta_1$ on
simply calls:
\small
\begin{Sinput}
>library(multcomp)